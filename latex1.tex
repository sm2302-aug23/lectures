\documentclass[,aspectratio=169]{beamer}

% ------------------------------------------------------------------------------
% Load theme -------------------------------------------------------------------
% ------------------------------------------------------------------------------
\usetheme[transitions]{ubd}

\newcommand{\prevarblock}{%
 \setbeamercolor{block title}{fg=ubdteal!30!ubdblack,bg=ubdteal!90}
 \setbeamercolor{block body}{parent=normal text,bg=ubdteal!10,fg=ubdteal!50!ubdblack}
}
\newenvironment{varblock}[1]{%
  \prevarblock
  \vspace{0.5em}
  \begin{block}{#1}}{\end{block}}

% Information for the title page -----------------------------------------------
\author{Drs. Haziq Jamil \& Huda Ramli}

\title{SM-2302 Software for Mathematicians}

\title{SM-2302 Software for Mathematicians}

\subtitle{\LaTeX 1: The basics \only<handout>{\emph{[handout version]}}}

\institute{Mathematical Sciences, Faculty of Science, UBD\\
\url{https://github.com/sm2302-aug23}}

\date{Semester I 2023/24}

% ------------------------------------------------------------------------------
% knitr stuff ------------------------------------------------------------------
% ------------------------------------------------------------------------------
\usepackage{color}
\usepackage{fancyvrb}
\newcommand{\VerbBar}{|}
\newcommand{\VERB}{\Verb[commandchars=\\\{\}]}
\DefineVerbatimEnvironment{Highlighting}{Verbatim}{commandchars=\\\{\}}
% Add ',fontsize=\small' for more characters per line
\usepackage{framed}
\definecolor{shadecolor}{HTML}{F5F5F5}
\newenvironment{Shaded}{\begin{snugshade}}{\end{snugshade}}
\newcommand{\AlertTok}[1]{\textcolor[rgb]{0.94,0.16,0.16}{#1}}
\newcommand{\AnnotationTok}[1]{\textcolor[rgb]{0.56,0.35,0.01}{\textbf{\textit{#1}}}}
\newcommand{\AttributeTok}[1]{\textcolor[rgb]{0.77,0.63,0.00}{#1}}
\newcommand{\BaseNTok}[1]{\textcolor[rgb]{0.00,0.00,0.81}{#1}}
\newcommand{\BuiltInTok}[1]{#1}
\newcommand{\CharTok}[1]{\textcolor[rgb]{0.31,0.60,0.02}{#1}}
\newcommand{\CommentTok}[1]{\textcolor[rgb]{0.56,0.35,0.01}{\textit{#1}}}
\newcommand{\CommentVarTok}[1]{\textcolor[rgb]{0.56,0.35,0.01}{\textbf{\textit{#1}}}}
\newcommand{\ConstantTok}[1]{\textcolor[rgb]{0.00,0.00,0.00}{#1}}
\newcommand{\ControlFlowTok}[1]{\textcolor[rgb]{0.13,0.29,0.53}{\textbf{#1}}}
\newcommand{\DataTypeTok}[1]{\textcolor[rgb]{0.13,0.29,0.53}{#1}}
\newcommand{\DecValTok}[1]{\textcolor[rgb]{0.00,0.00,0.81}{#1}}
\newcommand{\DocumentationTok}[1]{\textcolor[rgb]{0.56,0.35,0.01}{\textbf{\textit{#1}}}}
\newcommand{\ErrorTok}[1]{\textcolor[rgb]{0.64,0.00,0.00}{\textbf{#1}}}
\newcommand{\ExtensionTok}[1]{#1}
\newcommand{\FloatTok}[1]{\textcolor[rgb]{0.00,0.00,0.81}{#1}}
\newcommand{\FunctionTok}[1]{\textcolor[rgb]{0.00,0.00,0.00}{#1}}
\newcommand{\ImportTok}[1]{#1}
\newcommand{\InformationTok}[1]{\textcolor[rgb]{0.56,0.35,0.01}{\textbf{\textit{#1}}}}
\newcommand{\KeywordTok}[1]{\textcolor[rgb]{0.13,0.29,0.53}{\textbf{#1}}}
\newcommand{\NormalTok}[1]{#1}
\newcommand{\OperatorTok}[1]{\textcolor[rgb]{0.81,0.36,0.00}{\textbf{#1}}}
\newcommand{\OtherTok}[1]{\textcolor[rgb]{0.56,0.35,0.01}{#1}}
\newcommand{\PreprocessorTok}[1]{\textcolor[rgb]{0.56,0.35,0.01}{\textit{#1}}}
\newcommand{\RegionMarkerTok}[1]{#1}
\newcommand{\SpecialCharTok}[1]{\textcolor[rgb]{0.00,0.00,0.00}{#1}}
\newcommand{\SpecialStringTok}[1]{\textcolor[rgb]{0.31,0.60,0.02}{#1}}
\newcommand{\StringTok}[1]{\textcolor[rgb]{0.31,0.60,0.02}{#1}}
\newcommand{\VariableTok}[1]{\textcolor[rgb]{0.00,0.00,0.00}{#1}}
\newcommand{\VerbatimStringTok}[1]{\textcolor[rgb]{0.31,0.60,0.02}{#1}}
\newcommand{\WarningTok}[1]{\textcolor[rgb]{0.56,0.35,0.01}{\textbf{\textit{#1}}}}
\usepackage{graphicx,grffile}
\makeatletter
\def\maxwidth{\ifdim\Gin@nat@width>\linewidth\linewidth\else\Gin@nat@width\fi}
\def\maxheight{\ifdim\Gin@nat@height>\textheight\textheight\else\Gin@nat@height\fi}
\makeatother
% Scale images if necessary, so that they will not overflow the page
% margins by default, and it is still possible to overwrite the defaults
% using explicit options in \includegraphics[width, height, ...]{}
\setkeys{Gin}{width=\maxwidth,height=\maxheight,keepaspectratio}
% Set default figure placement to htbp
\makeatletter
\def\fps@figure{htbp}
\makeatother
\setlength{\emergencystretch}{3em} % prevent overfull lines
\providecommand{\tightlist}{%
  \setlength{\itemsep}{0pt}\setlength{\parskip}{0pt}}
\setcounter{secnumdepth}{-\maxdimen} % remove section numbering

% ------------------------------------------------------------------------------
% Packages ---------------------------------------------------------------------
% ------------------------------------------------------------------------------

\usepackage{empheq}
\usepackage{ragged2e}
\usepackage{tikz}
\usetikzlibrary{shapes.geometric,fit,arrows.meta}
\usepackage{xltabular,longtable,booktabs,multirow,multicol,colortbl}

% \usepackage{caption}
% % Make caption package work with longtable
% \makeatletter
% \def\fnum@table{\tablename~\thetable}
% \makeatother

% highlight stuff using \hlc
\usepackage{soul}
\sethlcolor{ubdyellow}
\makeatletter
\let\HL\hl
\renewcommand\hl{%
  \let\set@color\beamerorig@set@color
  \let\reset@color\beamerorig@reset@color
  \HL}
\makeatother
% https://tex.stackexchange.com/questions/460731/highlight-color-a-part-of-text-in-block-in-beamer
\newcommand{\hlc}[2][ubdyellow]{{%
    \colorlet{foo}{#1}%
    \sethlcolor{foo}\hl{#2}}%
}
% https://tex.stackexchange.com/questions/352956/how-to-highlight-text-with-an-arbitrary-color

% To use arabic ----------------------------------------------------------------
% WARNING: Using arabic script causes some issues with footnotes.
% Packages are not loaded by default
% \usepackage{polyglossia}  
% \setdefaultlanguage{english}
% \setotherlanguage{arabic} % to use arabic
% \newfontfamily\arabicfontsf[Script=Arabic]{Amiri}
% % Note that when arabic is set, the itemize becomes triangles
% \setbeamertemplate{itemize item}[circ]
% \setbeamertemplate{itemize subitem}[circ]
% \setbeamertemplate{itemize subsubitem}[circ]
% \usepackage{xeCJK}
% \setCJKmainfont{SimSun}
% \setCJKsansfont{FangSong}
% \setCJKmonofont{KaiTi}

% % Fix for footnotes not showing when arabic script used
% % https://tex.stackexchange.com/questions/228075/beamer-in-arabic-language-doesnt-accept-footnotes
% \makeatletter
% \let\@footnotetext=\beamer@framefootnotetext
% \makeatother

% % Fix for footnotes not showing when using \footnote<.->
% \let\oldfootnote\footnote
% \renewcommand{\footnote}{\only<+->\oldfootnote}
% % https://stackoverflow.com/questions/62345074/show-footnote-only-after-a-pause-in-beamer-with-r-markdown

\usepackage{polyglossia}
\setdefaultlanguage{english}
\setotherlanguage{arabic} % to use arabic
\newfontfamily\arabicfontsf[Script=Arabic]{Amiri}
\usepackage{xeCJK}
\setCJKmainfont{SimSun}
\setbeamertemplate{itemize item}[circ]
\setbeamertemplate{itemize subitem}[circ]
\setbeamertemplate{itemize subsubitem}[circ]
\usepackage{chemfig}
\setchemfig{atom sep = 2em, bond join = true}
\usepackage{chemmacros}
\chemsetup{modules = {all}}
\setbeamertemplate{caption}[numbered]
\usepackage{tikz}
\usetikzlibrary{shapes,arrows,positioning,shadows}

% Bibliography -----------------------------------------------------------------
\usepackage[,maxcitenames=2,maxbibnames=99,backend=biber,natbib]{biblatex}
\renewcommand*{\mkbibacro}[1]{#1}  % fix URL, DOI, ISBN, etc. font
\addbibresource{refs.bib}

% ------------------------------------------------------------------------------
% Mathematics ------------------------------------------------------------------
% ------------------------------------------------------------------------------

% coloured box around theorems etc.
\usepackage[skins,theorems]{tcolorbox}
\tcbset{highlight math style={enhanced,
  colframe=ubdblue,colback=white,arc=2pt,boxrule=1pt}}

\usepackage{pifont}
\newcommand{\cmark}{\ding{51}}%
\newcommand{\xmark}{\ding{55}}%

\usepackage{amsmath,amssymb}
\usepackage{dsfont}  % for indicator variables \mathsds{1}
\usepackage{bm}  % for better bold script
\usepackage[makeroom]{cancel}
\usepackage{centernot}
\renewcommand{\CancelColor}{\color{gray}}
\newcommand{\bzero}{{\bm 0}}
\newcommand{\bone}{{\bm 1}}
\newcommand{\baa}{{\bm a}}
\newcommand{\bb}{{\bm b}}
\newcommand{\bc}{{\bm c}}
\newcommand{\bd}{{\bm d}}
\newcommand{\be}{{\bm e}}
\newcommand{\bff}{{\bm f}}
\newcommand{\bg}{{\bm g}}
\newcommand{\bh}{{\bm h}}
\newcommand{\bi}{{\bm i}}
\newcommand{\bj}{{\bm j}}
\newcommand{\bk}{{\bm k}}
\newcommand{\bl}{{\bm l}}
\newcommand{\bmm}{{\bm m}}
\newcommand{\bn}{{\bm n}}
\newcommand{\bo}{{\bm o}}
\newcommand{\bp}{{\bm p}}
\newcommand{\bq}{{\bm q}}
\newcommand{\br}{{\bm r}}
\newcommand{\bs}{{\bm s}}
\newcommand{\bt}{{\bm t}}
\newcommand{\bu}{{\bm u}}
\newcommand{\bv}{{\bm v}}
\newcommand{\bw}{{\bm w}}
\newcommand{\bx}{{\bm x}}
\newcommand{\by}{{\bm y}}
\newcommand{\bz}{{\bm z}}
\newcommand{\bA}{{\bm A}}
\newcommand{\bB}{{\bm B}}
\newcommand{\bC}{{\bm C}}
\newcommand{\bD}{{\bm D}}
\newcommand{\bE}{{\bm E}}
\newcommand{\bF}{{\bm F}}
\newcommand{\bG}{{\bm G}}
\newcommand{\bH}{{\bm H}}
\newcommand{\bI}{{\bm I}}
\newcommand{\bJ}{{\bm J}}
\newcommand{\bK}{{\bm K}}
\newcommand{\bL}{{\bm L}}
\newcommand{\bM}{{\bm M}}
\newcommand{\bN}{{\bm N}}
\newcommand{\bO}{{\bm O}}
\newcommand{\bP}{{\bm P}}
\newcommand{\bQ}{{\bm Q}}
\newcommand{\bR}{{\bm R}}
\newcommand{\bS}{{\bm S}}
\newcommand{\bT}{{\bm T}}
\newcommand{\bU}{{\bm U}}
\newcommand{\bV}{{\bm V}}
\newcommand{\bW}{{\bm W}}
\newcommand{\bX}{{\bm X}}
\newcommand{\bY}{{\bm Y}}
\newcommand{\bZ}{{\bm Z}}

% Greek bold letters
\newcommand{\balpha}{{\bm\alpha}}
\newcommand{\bbeta}{{\bm\beta}}
\newcommand{\bgamma}{{\bm\gamma}}
\newcommand{\bdelta}{{\bm\delta}}
\newcommand{\bepsilon}{{\bm\epsilon}}
\newcommand{\bvarepsilon}{{\bm\varepsilon}}
\newcommand{\bzeta}{{\bm\zeta}}
\newcommand{\bfeta}{{\bm\eta}}
\newcommand{\boldeta}{{\bm\eta}}
\newcommand{\btheta}{{\bm\theta}}
\newcommand{\bvartheta}{{\bm\vartheta}}
\newcommand{\biota}{{\bm\iota}}
\newcommand{\bkappa}{{\bm\kappa}}
\newcommand{\blambda}{{\bm\lambda}}
\newcommand{\bmu}{{\bm\mu}}
\newcommand{\bnu}{{\bm\nu}}
\newcommand{\bxi}{{\bm\xi}}
\newcommand{\bpi}{{\bm\pi}}
\newcommand{\bvarpi}{{\bm\varpi}}
\newcommand{\brho}{{\bm\rho}}
\newcommand{\bvarrho}{{\bm\varrho}}
\newcommand{\bsigma}{{\bm\sigma}}
\newcommand{\bvarsigma}{{\bm\varsigma}}
\newcommand{\btau}{{\bm\tau}}
\newcommand{\bupsilon}{{\bm\upsilon}}
\newcommand{\bphi}{{\bm\phi}}
\newcommand{\bvarphi}{{\bm\varphi}}
\newcommand{\bchi}{{\bm\chi}}
\newcommand{\bpsi}{{\bm\psi}}
\newcommand{\bomega}{{\bm\omega}}

\newcommand{\bGamma}{{\bm\Gamma}}
\newcommand{\bDelta}{{\bm\Delta}}
\newcommand{\bTheta}{{\bm\Theta}}
\newcommand{\bLambda}{{\bm\Lambda}}
\newcommand{\bXi}{{\bm\Xi}}
\newcommand{\bPi}{{\bm\Pi}}
\newcommand{\bSigma}{{\bm\Sigma}}
\newcommand{\bUpsilon}{{\bm\Upsilon}}
\newcommand{\bPhi}{{\bm\Phi}}
\newcommand{\bPsi}{{\bm\Psi}}
\newcommand{\bOmega}{{\bm\Omega}}

% Probability and Statistics
\DeclareMathOperator{\Prob}{P}
\DeclareMathOperator{\E}{E}
\DeclareMathOperator{\Var}{Var}
\DeclareMathOperator{\Cov}{Cov}
\DeclareMathOperator{\Corr}{Corr}
\DeclareMathOperator{\sd}{sd}
\DeclareMathOperator{\se}{se}
\DeclareMathOperator{\N}{N}
\DeclareMathOperator{\Bin}{Bin}
\DeclareMathOperator{\Bern}{Bern}
\DeclareMathOperator{\Dir}{Dir}
\DeclareMathOperator{\Wis}{Wis}
\DeclareMathOperator{\logit}{logit}
\DeclareMathOperator{\expit}{expit}
\DeclareMathOperator{\Mult}{Mult}
\DeclareMathOperator{\Cat}{Cat}
\DeclareMathOperator{\Pois}{Poi}
\DeclareMathOperator{\Geom}{Geom}
\DeclareMathOperator{\NBin}{NBin}
\DeclareMathOperator{\Exp}{Exp}
\DeclareMathOperator{\Betadist}{Beta}
\DeclareMathOperator{\Hypergeom}{Hypergeom}
\DeclareMathOperator{\Cauchy}{Cauchy}
\DeclareMathOperator{\hCauchy}{half-Cauchy}
\DeclareMathOperator{\LKJ}{LKJ}
\DeclareMathOperator{\Unif}{Unif}
\DeclareMathOperator{\KL}{KL}
\DeclareMathOperator{\ind}{\mathds{1}}
\newcommand{\iid}{\,\overset{\text{iid}}{\sim}\,}
\DeclareMathOperator*{\plim}{plim}
\DeclareMathOperator{\Lik}{L}
\DeclareMathOperator{\Leb}{Leb}

% Blackboard bold
\newcommand{\bbR}{\mathbb{R}}
\newcommand{\bbN}{\mathbb{N}}
\newcommand{\bbZ}{\mathbb{Z}}
\newcommand{\bbC}{\mathbb{C}}
\newcommand{\bbS}{\mathbb{S}}
\newcommand{\bbH}{\mathbb{H}}
\newcommand{\bbP}{\mathbb{P}}
\newcommand{\bbQ}{\mathbb{Q}}
\newcommand{\bbE}{\mathbb{E}}

% Math calligraphic fonts
\newcommand{\cA}{{\mathcal A}}
\newcommand{\cB}{{\mathcal B}}
\newcommand{\cC}{{\mathcal C}}
\newcommand{\cD}{{\mathcal D}}
\newcommand{\cE}{{\mathcal E}}
\newcommand{\cF}{{\mathcal F}}
\newcommand{\cG}{{\mathcal G}}
\newcommand{\cH}{{\mathcal H}}
\newcommand{\cI}{{\mathcal I}}
\newcommand{\cJ}{{\mathcal J}}
\newcommand{\cK}{{\mathcal K}}
\newcommand{\cL}{{\mathcal L}}
\newcommand{\cM}{{\mathcal M}}
\newcommand{\cN}{{\mathcal N}}
\newcommand{\cO}{{\mathcal O}}
\newcommand{\cP}{{\mathcal P}}
\newcommand{\cQ}{{\mathcal Q}}
\newcommand{\cR}{{\mathcal R}}
\newcommand{\cS}{{\mathcal S}}
\newcommand{\cT}{{\mathcal T}}
\newcommand{\cU}{{\mathcal U}}
\newcommand{\cV}{{\mathcal V}}
\newcommand{\cW}{{\mathcal W}}
\newcommand{\cX}{{\mathcal X}}
\newcommand{\cY}{{\mathcal Y}}
\newcommand{\cZ}{{\mathcal Z}}

% Overbrace and underbrace
\newcommand{\myoverbrace}[3][gray!70]{{\color{#1}\overbrace{\color{black}#2}^{#3}}}
\newcommand{\myunderbrace}[3][gray!70]{{\color{#1}\underbrace{\color{black}#2}_{#3}}}

% Conveniences
\newcommand{\const}{\text{const.}}
\newcommand{\half}[1][1]{\frac{#1}{2}}  % \half for 1/2 or \half[n] for n/2, etc.
\DeclareMathOperator{\diag}{diag}
\DeclareMathOperator{\tr}{tr}
\DeclareMathOperator*{\argmin}{arg\,min}
\DeclareMathOperator*{\argmax}{arg\,max}

% Comments gray text
\newcommand{\mycomment}[2][10pt]{\hspace{#1}\rlap{\color{gray}\text{#2}}}

% Derivatives and integration
\let\d\relax
\DeclareMathOperator{\dd}{d}
\newcommand{\dint}{\dd\hspace{0.5pt}\!}
\newcommand{\d}{\text{d}}

% https://tex.stackexchange.com/questions/19981/how-to-write-rudins-symbol-for-absolute-continuity-of-measures
\DeclareFontFamily{U}{matha}{\hyphenchar\font45}
\DeclareFontShape{U}{matha}{m}{n}{
  <-6> matha5 <6-7> matha6 <7-8> matha7
  <8-9> matha8 <9-10> matha9
  <10-12> matha10 <12-> matha12
  }{}
\DeclareSymbolFont{matha}{U}{matha}{m}{n}
\DeclareMathSymbol{\Lt}{3}{matha}{"CE}

\graphicspath{ {figure/} }

\begin{document}

\begin{frame}[plain,noframenumbering]
	\titlepage
\end{frame}


\hypertarget{introduction}{%
\section{Introduction}\label{introduction}}

\begin{frame}{Why \LaTeX?}
\protect\hypertarget{why}{}
\vspace{-0.5em}

\footnotesize

\begin{center}\includegraphics[width=1\linewidth,height=0.5\textheight]{figure/kerning} \end{center}

\normalsize

\vspace{-0.5em}

\begin{itemize}
\item
  It makes \href{https://nitens.org/w/latex/}{\underline{beautiful}}
  documents (kerning, ligatures,
  \href{https://tex.stackexchange.com/questions/110133/visual-comparison-between-latex-and-word-output-hyphenation-typesetting-ligat}{\underline{hyphenation}}).
\item
  Open source and active community. Lots of packages available.
\item
  Extensible document types (articles, presentation slides, books,
  theses, exam papers, etc.).
\end{itemize}

\begin{alertblock}{Reminder}
Sign up for Overleaf if you haven't done so!

\end{alertblock}
\end{frame}

\hypertarget{how-does-it-work}{%
\subsection{How does it work?}\label{how-does-it-work}}

\begin{frame}[fragile]{How does it work?}
\begin{itemize}
\item
  You write your document in \texttt{plain\ text} with commands that
  describe its structure and meaning.
\item
  The \LaTeX~program then processes your text and commands to produce a
  beautifully formatted document.
\end{itemize}

\begin{Shaded}
\begin{Highlighting}[]
\NormalTok{The rain in Spain falls }\FunctionTok{\textbackslash{}emph}\NormalTok{\{mainly\} on the plain.}
\end{Highlighting}
\end{Shaded}

\centering
\vskip 1ex
\tikz\node[single arrow,fill=gray,font=\ttfamily\bfseries,%
  rotate=270,xshift=-1em]{latex};
\vskip 1ex
\begin{center}\Large
The rain in Spain falls \emph{mainly} on the plain.
\end{center}

\blfootnote{This workshop is inspired by the \LaTeX\ course by 
\href{https://github.com/jdleesmiller/latex-course}{JD Miller}. MIT license.}
\end{frame}

\begin{frame}[fragile]{More examples of commands and output\ldots{}}
\protect\hypertarget{more-examples-of-commands-and-output}{}
\vspace{-1.5em}

\begin{columns}[T]
\begin{column}{0.48\textwidth}
\begin{Shaded}
\begin{Highlighting}[]
\KeywordTok{\textbackslash{}begin}\NormalTok{\{}\ExtensionTok{itemize}\NormalTok{\}}
  \FunctionTok{\textbackslash{}item}\NormalTok{ Tea}
  \FunctionTok{\textbackslash{}item}\NormalTok{ Milk}
  \FunctionTok{\textbackslash{}item}\NormalTok{ Biscuits}
\KeywordTok{\textbackslash{}end}\NormalTok{\{}\ExtensionTok{itemize}\NormalTok{\}}
\end{Highlighting}
\end{Shaded}
\end{column}

\begin{column}{0.48\textwidth}
\vspace{2em}

\begin{itemize}
\item Tea
\item Milk
\item Biscuits
\end{itemize}
\end{column}
\end{columns}

\begin{columns}[T]
\begin{column}{0.48\textwidth}
\begin{Shaded}
\begin{Highlighting}[]
\KeywordTok{\textbackslash{}begin}\NormalTok{\{}\ExtensionTok{figure}\NormalTok{\}}
  \BuiltInTok{\textbackslash{}includegraphics}\NormalTok{\{}\ExtensionTok{gerbil}\NormalTok{\}}
\KeywordTok{\textbackslash{}end}\NormalTok{\{}\ExtensionTok{figure}\NormalTok{\}}
\end{Highlighting}
\end{Shaded}
\end{column}

\begin{column}{0.48\textwidth}
\begin{figure}
\includegraphics{gerbil}*
\end{figure}
\end{column}
\end{columns}

\begin{columns}[T]
\begin{column}{0.48\textwidth}
\begin{Shaded}
\begin{Highlighting}[]
\KeywordTok{\textbackslash{}begin}\NormalTok{\{}\ExtensionTok{equation}\NormalTok{\}}
\SpecialStringTok{y = }\SpecialCharTok{\textbackslash{}alpha}\SpecialStringTok{ + }\SpecialCharTok{\textbackslash{}beta}\SpecialStringTok{ x}
\KeywordTok{\textbackslash{}end}\NormalTok{\{}\ExtensionTok{equation}\NormalTok{\}}
\end{Highlighting}
\end{Shaded}
\end{column}

\begin{column}{0.48\textwidth}
\vspace{2.5em}

\begin{equation}
y = \alpha + \beta x
\end{equation}
\end{column}
\end{columns}

\blfootnote{*Image license: \href{https://pixabay.com/en/animal-apple-attractive-beautiful-1239390/}{CC0}}
\end{frame}

\begin{frame}{Attitude adjustment}
\protect\hypertarget{attitude-adjustment}{}
\begin{itemize}
\item
  Use commands to describe `what it is' and not `how it looks'.
\item
  Focus on your content.
\item
  Let \LaTeX~do its job.
\end{itemize}

\footnotesize

\begin{center}\includegraphics[width=1\linewidth]{figure/latexgraph-1} \end{center}

\normalsize
\end{frame}

\hypertarget{showcasing}{%
\subsection{\texorpdfstring{Showcasing
\LaTeX}{Showcasing }}\label{showcasing}}

\begin{frame}{Float placements}
\protect\hypertarget{float-placements}{}
\LaTeX~takes care of figure placements (``floats'') automatically.

\vspace{-1em}

\begin{columns}[T]
\begin{column}{0.64\textwidth}
\footnotesize

\begin{center}\includegraphics[width=1\linewidth]{figure/piechart-1} \end{center}

\normalsize
\end{column}

\begin{column}{0.33\textwidth}
\vspace{2em}

\begin{figure}
\frame{\includegraphics[width=\linewidth]{figure/tweet_msword.pdf}}
\end{figure}
\end{column}
\end{columns}
\end{frame}

\begin{frame}{Citations}
\protect\hypertarget{citations}{}
\justifying

Sometimes, however, what others tell us is important as
\emph{corroboration} of what we have already found out (or think we have
found out) for ourselves. The Scottish philosopher Thomas Reid makes
this point in connection with mathematical research in the belief that,
if it applies to the science `in which, of all sciences, authority is
acknowledged to have least weight' \cite{reid2002thomas}, it will be
even more significant in other areas of thought and practice\ldots
Russell, as we shall see in a later chapter, considered this aspect of
our reliance upon testimony essential to the understanding of what it is
to be a physical thing and he criticized logical positivism for its
failure to appreciate the implications of this point
\cite{russell2007logic}. In the Analysis of Matter he says explicitly,
`I mean here by ``objective'' not anything metaphysical but merely
``agreeing with the testimony of others''\,' \cite{russell2015analysis}.

\vspace{1em}

Excerpt from \emph{Testimony: A Philosophical Study} by C. A. J. Coady
(1992)
\end{frame}

\begin{frame}{Bibliography}
\protect\hypertarget{bibliography}{}
\nocite{coady1992testimony} \printbibliography[heading=none]
\end{frame}

\begin{frame}{Mathematics}
\protect\hypertarget{mathematics}{}
For \(i=1,\dots,n\), let \begin{equation}\label{mod1}
\begin{gathered}
y_i = f(x_i) + \epsilon_i \\
(\epsilon_1,\dots,\epsilon_n)^\top \sim \N_n(0, \Psi^{-1}),
\end{gathered}
\end{equation} where \(y_i\in\bbR\), \(x_i\in\mathcal X\), and
\(f\in\mathcal F\) a reproducing kernel Hilbert space (RKHS) of
functions with kernel \(h:\mathcal X \times \mathcal X \to \mathbb R\).

\begin{lemma}[Fisher information for regression function]
\label{thm:fisherinformation} For the normal model \eqref{mod1} with
log-likelihood \(\ell\), the Fisher information for \(f\) is
\begin{equation}
\cI_f = -\E\nabla^2 \ell(f|y) = \sum_{i=1}^n\sum_{j=1}^n \psi_{ij}h(\cdot,x_i) \otimes h(\cdot,x_j) \label{eq:fisherinfo}
\end{equation} where `\(\otimes\)' is the tensor product of two vectors
in \(\mathcal F\).
\end{lemma}

The bilinear form \eqref{eq:fisherinfo} in Lemma
\ref{thm:fisherinformation} is a consequence of variational calculus.
\end{frame}

\begin{frame}{Chemical equations}
\protect\hypertarget{chemical-equations}{}
\begin{figure}
    \centering
    \small
    \fbox{
    \schemestart
        \chemname{
            \chemfig{
                CH
                (-[:90]CH_2-OOC-R_1)
                (-[:-90]CH_2-OOC-R_3)
                -OOC-R_2
            }
        }{Triglyceride}
        \+
        \chemname{
                \chemfig{
                    3 ROH
                }
            }{Alcohol}
        \arrow(.mid east--.mid west){<=>[Catalyst]}
        \chemname{
            \chemfig{
                R_2
                (-[:90,,,,draw=none]R_1-COO-R)
                (-[:-90,,,,draw=none]R_3-COO-R)
                -COO-R
            }
        }{Alkyl esters}
        \+
        \chemname{
            \chemfig{
                CH
                (-[:90]CH_2-OH)
                (-[:-90]CH_2-OH)
                -OH
            }
        }{Glycerol}
    \schemestop
    }
    \vspace{0.5em}
    \caption{Transesterification of triglyceride with alcohol.}
    \label{scm:tsester}
\end{figure}

\blfootnote{Figure \ref{scm:tsester} obtained from \url{https://tex.stackexchange.com/a/472486}}
\end{frame}

\begin{frame}{Multilingual support}
\protect\hypertarget{multilingual-support}{}
\vspace{-2em}

\begin{columns}[T]
\begin{column}{0.48\textwidth}
\justifying

\footnotesize

\begin{center}\includegraphics[width=1\linewidth,height=0.45\textheight]{figure/00-aljabr} \end{center}

\normalsize

\textarabic{
الْكِتَابْ الْمُخْتَصَرْ فِيْ حِسَابْ الْجَبْرْ وَالْمُقَابَلَة
} (The Compendious Book on Calculation by Completion and Balancing),
also known as \textarabic{الجبر} (Al-Jabr), written by \textarabic{
محمد بن موسى الخوارزميّ
} (Muḥammad ibn Mūsā al-Khwārizmī) around 820 CE.
\end{column}

\begin{column}{0.48\textwidth}
\justifying

\footnotesize

\begin{center}\includegraphics[width=1\linewidth,height=0.45\textheight]{figure/00-haidaosuanjing} \end{center}

\normalsize

海岛算经 (Hǎidǎo suàn jīng--The Sea Island Mathematical Manual) was
written by 刘徽 (Liú Huī) ca. 200 CE. The Chinese were aware of a good
approximation of \(\pi\approx\) \(355/113\) \(= 3.1415929204\) very
early on (祖冲之 Zǔ Chōng Zhī, 500 CE).
\end{column}
\end{columns}
\end{frame}

\hypertarget{getting-started}{%
\section{Getting started}\label{getting-started}}

\begin{frame}[fragile]{Getting started}
\framesubtitle{A minimal \LaTeX\ document}

\begin{Shaded}
\begin{Highlighting}[]
\BuiltInTok{\textbackslash{}documentclass}\NormalTok{\{}\ExtensionTok{article}\NormalTok{\}}
\KeywordTok{\textbackslash{}begin}\NormalTok{\{}\ExtensionTok{document}\NormalTok{\}}
\NormalTok{Hello, World!  }\CommentTok{\% your content goes here...}
\KeywordTok{\textbackslash{}end}\NormalTok{\{}\ExtensionTok{document}\NormalTok{\}}
\end{Highlighting}
\end{Shaded}

\begin{itemize}
\item
  Commands start with a backslash \framebox{\texttt{\textbackslash}}.
\item
  Every document starts with a
  \VERB|\BuiltInTok{\textbackslash{}documentclass}| command.
\item
  The \emph{argument} in curly braces \framebox{\texttt{\{}}
  \framebox{\texttt{\}}} tells \LaTeX~what kind of document we are
  creating (in this case, an \texttt{article}).
\item
  A percent sign \texttt{\%} starts a \emph{comment}--\LaTeX~will ignore
  the rest of the line.
\end{itemize}
\end{frame}

\begin{frame}{Getting started}
\protect\hypertarget{getting-started-1}{}
\vspace{-0.5em}

\footnotesize

\begin{center}\includegraphics[width=0.5\linewidth]{figure/overleaf} \end{center}

\normalsize

\begin{center}
\url{https://www.overleaf.com/}

\end{center}

\begin{itemize}
\item
  Overleaf is a website for writing documents in \LaTeX.
\item
  It `compiles' your \LaTeX~document online to show you the results.
\item
  As we go through the following slides, try out the examples by typing
  them into the example document on Overleaf!
\end{itemize}

\begin{varblock}{Exercise 0 (Hello world)}
Click
\href{https://www.overleaf.com/docs?snip_uri=https://raw.github.com//haziqj/learn-latex/main/exercises/00-hello_world/00-hello_world.tex&splash=none}{\beamerbutton{Hello World}}
to open the ``Hello world'' document in \textbf{Overleaf} (you'll need
to sign in first). Let's get started!

\end{varblock}
\end{frame}

\begin{frame}[fragile]{Typesetting text}
\protect\hypertarget{typesetting-text}{}
\begin{itemize}
\item
  Type your text between
  \VERB|\KeywordTok{\textbackslash{}begin}\NormalTok{\{}\ExtensionTok{document}\NormalTok{\}}|
  and
  \VERB|\KeywordTok{\textbackslash{}end}\NormalTok{\{}\ExtensionTok{document}\NormalTok{\}}|.
\item
  For the most part, you can just type your text normally.
\end{itemize}

\vspace{1em}

\begin{columns}[T]
\begin{column}{0.48\textwidth}
\vspace{-1em}

\begin{Shaded}
\begin{Highlighting}[]
\NormalTok{Words are separated by one}
\NormalTok{or more spaces.}

\NormalTok{Paragraphs are separated by}
\NormalTok{one or more blank lines.}
\end{Highlighting}
\end{Shaded}
\end{column}

\begin{column}{0.48\textwidth}
Words are separated by one or more spaces.

\vspace{1.2em}

Paragraphs are separated by one or more blank lines.
\end{column}
\end{columns}

\vspace{0.5em}

\begin{itemize}
\tightlist
\item
  Blank space in the source file is collaped in the output.
\end{itemize}

\begin{columns}[T]
\begin{column}{0.48\textwidth}
\vspace{-1em}

\begin{Shaded}
\begin{Highlighting}[]
\NormalTok{The      rain    in   Spain}
\NormalTok{falls mainly on the   plain.}
\end{Highlighting}
\end{Shaded}
\end{column}

\begin{column}{0.48\textwidth}
The rain in Spain falls mainly on the plain.
\end{column}
\end{columns}
\end{frame}

\begin{frame}[fragile]{Typesetting text (Caveats)}
\protect\hypertarget{typesetting-text-caveats}{}
\begin{itemize}
\tightlist
\item
  Quotation marks are a bit tricky: Use a backtick \framebox{\texttt{`}}
  on the left and an apostrophe \framebox{\texttt{'}} on the right.
\end{itemize}

\begin{columns}[T]
\begin{column}{0.48\textwidth}
\vspace{-1em}

\begin{Shaded}
\begin{Highlighting}[]
\NormalTok{Single quotes: \textasciigrave{}text\textquotesingle{}.}

\NormalTok{Double quotes: \textasciigrave{}\textasciigrave{}text\textquotesingle{}\textquotesingle{}.}
\end{Highlighting}
\end{Shaded}
\end{column}

\begin{column}{0.48\textwidth}
Single quotes: `text'.

\vspace{1.2em}

Double quotes: ``text''.
\end{column}
\end{columns}

\vspace{0.5em}

\begin{itemize}
\item
  Some common characters have special meanings in \LaTeX :

  \begin{itemize}
  \tightlist
  \item
    \framebox{\texttt{\%}} is used to comment text
  \item
    \framebox{\texttt{\#}} is used for macros definitions
  \item
    \framebox{\texttt{\&}} is used for alignment
  \item
    \framebox{\texttt{\$}} is used for maths
  \end{itemize}
\item
  If you just type these, you'll get an error. If you want one to appear
  in the output, you have to \emph{escape} it by preceding it with a
  backslash \framebox{\texttt{\textbackslash}}.
\end{itemize}

\begin{columns}[T]
\begin{column}{0.48\textwidth}
\vspace{-1em}

\begin{Shaded}
\begin{Highlighting}[]
\FunctionTok{\textbackslash{}$} \FunctionTok{\textbackslash{}\%} \FunctionTok{\textbackslash{}\&} \FunctionTok{\textbackslash{}\#}
\end{Highlighting}
\end{Shaded}
\end{column}

\begin{column}{0.48\textwidth}
\$ \% \& \#
\end{column}
\end{columns}
\end{frame}

\begin{frame}[fragile]{Handling errors}
\protect\hypertarget{handling-errors}{}
\begin{itemize}
\item
  \LaTeX~can get confused when it is trying to compile your document. If
  it does, it stops with an error, which you must fix before it will
  produce any output.
\item
  For example, if you misspell \VERB|\FunctionTok{\textbackslash{}emph}|
  as \VERB|\FunctionTok{\textbackslash{}meph}|, \LaTeX~will stop with an
  \texttt{undefined\ control\ sequence} error, because
  \VERB|\FunctionTok{\textbackslash{}meph}| is not one of the commands
  it knows.
\end{itemize}

\begin{block}{Advice on errors}

\begin{enumerate}
\item
  Don't panic! Errors happen. The error messages can give a clue as to
  what's wrong.
\item
  Fix them as soon as they arise--if what you just typed caused an
  error, you can start your debugging there.
\item
  If there are multiple errors, start with the first one--the cause may
  even be above it.
\end{enumerate}

\end{block}
\end{frame}

\begin{frame}{Exercise}
\protect\hypertarget{exercise}{}
\begin{varblock}{Exercise 1 (Typesetting Text)}
Typeset the following
paragraph\footnote{\url{http://en.wikipedia.org/wiki/Economy_of_the_United_States}}
in \LaTeX :

\begin{quote}
In March 2006, Congress raised that ceiling an additional \$0.79
trillion to \$8.97 trillion, which is approximately 68\% of GDP. As of
October 4, 2008, the ``Emergency Economic Stabilization Act of 2008''
raised the current debt ceiling to \$11.3 trillion.
\end{quote}

Click
\href{https://www.overleaf.com/docs?snip_uri=https://raw.github.com//haziqj/learn-latex/main/exercises/01-getting_started/01-getting_started.tex&splash=none}{\beamerbutton{Exercise 1}}
to open this exercise in \textbf{Overleaf}.

\end{varblock}

Watch out for

\begin{itemize}
\item
  characters with special meanings \framebox{\texttt{\%}}
  \framebox{\texttt{\#}} \framebox{\texttt{\&}} \framebox{\texttt{\$}}
\item
  typesetting quotation marks correctly.
\end{itemize}
\end{frame}

\hypertarget{mathematics-1}{%
\section{Mathematics}\label{mathematics-1}}

\hypertarget{inline-equations}{%
\subsection{Inline equations}\label{inline-equations}}

\begin{frame}[fragile]{Inline equations}
\begin{itemize}
\tightlist
\item
  Dollar signs \framebox{\texttt{\$}} are used to mark mathematics in
  text.
\end{itemize}

\begin{columns}[T]
\begin{column}{0.48\textwidth}
\small
\vspace{-1em}

\begin{Shaded}
\begin{Highlighting}[]
\CommentTok{\% not so good:}
\NormalTok{Let a and b be distinct positive}
\NormalTok{integers, and let c = a {-} b + 1.}

\CommentTok{\% much better:}
\NormalTok{Let }\SpecialStringTok{$a$}\NormalTok{ and }\SpecialStringTok{$b$}\NormalTok{ be distinct}
\NormalTok{positive integers, and let}
\SpecialStringTok{$c = a {-} b + 1$}\NormalTok{.}
\end{Highlighting}
\end{Shaded}
\end{column}

\begin{column}{0.48\textwidth}
\vspace{1em}

Let a and b be distinct positive integers, and let c = a - b + 1.

\hfill\break

Let \(a\) and \(b\) be distinct positive integers, and let
\(c = a - b + 1\).
\end{column}
\end{columns}

\vspace{0.5em}

\begin{itemize}
\item
  Always use dollar signs in pairs--one to \textbf{begin} and one to
  \textbf{end}.
\item
  \LaTeX~handles spacing automatically; it ignores your spaces.
\end{itemize}

\begin{columns}[T]
\begin{column}{0.48\textwidth}
\small
\vspace{-1.1em}

\begin{Shaded}
\begin{Highlighting}[]
\NormalTok{Let }\SpecialStringTok{$y=mx+c$}\NormalTok{ be }\FunctionTok{\textbackslash{}ldots}

\NormalTok{Let }\SpecialStringTok{$y = m x + c$}\NormalTok{ be }\FunctionTok{\textbackslash{}ldots}
\end{Highlighting}
\end{Shaded}
\end{column}

\begin{column}{0.48\textwidth}
Let \(y=mx+c\) be \ldots

\vspace{1.2em}

Let \(y = m x + c\) be \ldots
\end{column}
\end{columns}
\end{frame}

\begin{frame}[fragile]{More notation}
\protect\hypertarget{more-notation}{}
\begin{itemize}
\tightlist
\item
  Use caret/hat \framebox{\ \texttt{\^}} for superscripts and underscore
  \framebox{$\vphantom{a}$\texttt{\_}} for subscripts.
\end{itemize}

\begin{columns}[T]
\begin{column}{0.58\textwidth}
\small
\vspace{-1em}

\begin{Shaded}
\begin{Highlighting}[]
\SpecialStringTok{$y = c\_2 x\^{}2 + c\_1 x + c\_0$}
\end{Highlighting}
\end{Shaded}
\end{column}

\begin{column}{0.38\textwidth}
\(y = c_2 x^2 + c_1 x + c_0\)
\end{column}
\end{columns}

\vspace{0.5em}

\begin{itemize}
\tightlist
\item
  Use curly braces \framebox{\texttt{\{}} \framebox{\texttt{\}}} to
  group supers/sub scripts.
\end{itemize}

\begin{columns}[T]
\begin{column}{0.58\textwidth}
\small
\vspace{-1em}

\begin{Shaded}
\begin{Highlighting}[]
\CommentTok{\% oops!}
\SpecialStringTok{$F\_n = F\_n{-}1 + F\_n{-}2$}

\CommentTok{\% ok!}
\SpecialStringTok{$F\_n = F\_\{n{-}1\} + F\_\{n{-}2\}$}
\end{Highlighting}
\end{Shaded}
\end{column}

\begin{column}{0.38\textwidth}
\vspace{1em}

\(F_n = F_n-1 + F_n-2\)

\vspace{2em}

\(F_n = F_{n-1} + F_{n-2}\)
\end{column}
\end{columns}

\vspace{0.5em}

\begin{itemize}
\tightlist
\item
  There are commands for Greek letters and common notation.
\end{itemize}

\begin{columns}[T]
\begin{column}{0.58\textwidth}
\small
\vspace{-1em}

\begin{Shaded}
\begin{Highlighting}[]
\SpecialStringTok{$}\SpecialCharTok{\textbackslash{}mu}\SpecialStringTok{ = A e\^{}\{Q/RT\}$}

\SpecialStringTok{$}\SpecialCharTok{\textbackslash{}Omega}\SpecialStringTok{ = }\SpecialCharTok{\textbackslash{}sum}\SpecialStringTok{\_\{k=1\}\^{}\{n\} }\SpecialCharTok{\textbackslash{}omega}\SpecialStringTok{\_k$}
\end{Highlighting}
\end{Shaded}
\end{column}

\begin{column}{0.38\textwidth}
\(\mu = A e^{Q/RT}\)

\vspace{1em}

\(\Omega = \sum_{k=1}^{n} \omega_k\)
\end{column}
\end{columns}
\end{frame}

\begin{frame}{Detexify}
\protect\hypertarget{detexify}{}
\centering
\vspace{-0.5em}

\href{https://detexify.kirelabs.org/classify.html}{\includegraphics{figure/detexify.png}}
\end{frame}

\hypertarget{displayed-equations}{%
\subsection{Displayed equations}\label{displayed-equations}}

\begin{frame}[fragile]{Displayed equations}
\begin{itemize}
\tightlist
\item
  If the mathematics is big and scary, \emph{display} it on its own line
  using
  \VERB|\KeywordTok{\textbackslash{}begin}\NormalTok{\{}\ExtensionTok{equation}\NormalTok{\}}|
  and
  \VERB|\KeywordTok{\textbackslash{}end}\NormalTok{\{}\ExtensionTok{equation}\NormalTok{\}}|
\end{itemize}

\begin{columns}[T]
\begin{column}{0.58\textwidth}
\small
\vspace{-1.1em}

\begin{Shaded}
\begin{Highlighting}[]
\NormalTok{The roots of a quadratic equation are}
\NormalTok{given by}
\KeywordTok{\textbackslash{}begin}\NormalTok{\{}\ExtensionTok{equation}\NormalTok{\}}
\SpecialStringTok{x = }\SpecialCharTok{\textbackslash{}frac}\SpecialStringTok{\{{-}b }\SpecialCharTok{\textbackslash{}pm}\SpecialStringTok{ }\SpecialCharTok{\textbackslash{}sqrt}\SpecialStringTok{\{b\^{}2 {-} 4ac\}\}}
\SpecialStringTok{         \{2a\}}
\KeywordTok{\textbackslash{}end}\NormalTok{\{}\ExtensionTok{equation}\NormalTok{\}}
\NormalTok{where }\SpecialStringTok{$a$}\NormalTok{, }\SpecialStringTok{$b$}\NormalTok{ and }\SpecialStringTok{$c$}\NormalTok{ are }\FunctionTok{\textbackslash{}ldots}
\end{Highlighting}
\end{Shaded}
\end{column}

\begin{column}{0.38\textwidth}
The roots of a quadratic equation are given by \begin{equation}
x = \frac{-b \pm \sqrt{b^2 - 4ac}}
{2a} \end{equation} where \(a\), \(b\) and \(c\) are \ldots
\end{column}
\end{columns}

\vspace{0.5em}

\begin{alertblock}{Caution}
\LaTeX~mostly ignores your spaces in mathematics, but it can't handle
blank lines in equations--don't put blank lines in your mathematics.

\end{alertblock}
\end{frame}

\hypertarget{interlude-environments}{%
\subsection{Interlude: Environments}\label{interlude-environments}}

\begin{frame}[fragile]{Interlude: Environments}
\begin{itemize}
\item
  \texttt{equation} is an \emph{environment} (a context).
\item
  The \VERB|\KeywordTok{\textbackslash{}begin}| and
  \VERB|\KeywordTok{\textbackslash{}end}| commands are used to create
  many different environments. E.g., \texttt{itemize} and
  \texttt{enumerate} for lists:
\end{itemize}

\begin{columns}[T]
\begin{column}{0.58\textwidth}
\small
\vspace{-1em}

\begin{Shaded}
\begin{Highlighting}[]
\KeywordTok{\textbackslash{}begin}\NormalTok{\{}\ExtensionTok{itemize}\NormalTok{\} }\CommentTok{\% for bullet points}
\FunctionTok{\textbackslash{}item}\NormalTok{ Biscuits}
\FunctionTok{\textbackslash{}item}\NormalTok{ Tea}
\KeywordTok{\textbackslash{}end}\NormalTok{\{}\ExtensionTok{itemize}\NormalTok{\}}

\KeywordTok{\textbackslash{}begin}\NormalTok{\{}\ExtensionTok{enumerate}\NormalTok{\} }\CommentTok{\% for numbers}
\FunctionTok{\textbackslash{}item}\NormalTok{ Biscuits}
\FunctionTok{\textbackslash{}item}\NormalTok{ Tea}
\KeywordTok{\textbackslash{}end}\NormalTok{\{}\ExtensionTok{enumerate}\NormalTok{\}}
\end{Highlighting}
\end{Shaded}
\end{column}

\begin{column}{0.38\textwidth}
\vspace{1em}

\begin{itemize}
\tightlist
\item
  Biscuits
\item
  Tea
\end{itemize}

\vspace{2.5em}

\begin{enumerate}
\tightlist
\item
  Biscuits
\item
  Tea
\end{enumerate}
\end{column}
\end{columns}
\end{frame}

\hypertarget{interlude-packages}{%
\subsection{Interlude: Packages}\label{interlude-packages}}

\begin{frame}[fragile]{Interlude: Packages}
\begin{itemize}
\item
  All of the commands and environments we've used so far are built into
  \LaTeX .
\item
  \emph{Packages} are libraries of extra commands and evironments. There
  are thousands of freely available packages.
\item
  We have to load each package we want to use with a
  \VERB|\BuiltInTok{\textbackslash{}usepackage}| commands in the
  \emph{preamble}.
\item
  Example: \texttt{amsmath} from the American Mathematical Society.
\end{itemize}

\begin{Shaded}
\begin{Highlighting}[]
\BuiltInTok{\textbackslash{}documentclass}\NormalTok{\{}\ExtensionTok{article}\NormalTok{\}}
\BuiltInTok{\textbackslash{}usepackage}\NormalTok{\{}\ExtensionTok{amsmath}\NormalTok{\} }\CommentTok{\% preamble}
\KeywordTok{\textbackslash{}begin}\NormalTok{\{}\ExtensionTok{document}\NormalTok{\}}
\CommentTok{\% now we can use commands from amsmath here...}
\KeywordTok{\textbackslash{}end}\NormalTok{\{}\ExtensionTok{document}\NormalTok{\}}
\end{Highlighting}
\end{Shaded}
\end{frame}

\begin{frame}[fragile]{An example with \texttt{amsmath}}
\protect\hypertarget{an-example-with-amsmath}{}
\begin{itemize}
\tightlist
\item
  Align a sequence of equations at the equals sign \begin{align*}
  (x+1)^3 &= (x+1)(x+1)(x+1) \\
        &= (x+1)(x^2 + 2x + 1) \\
        &= x^3 + 3x^2 + 3x + 1
  \end{align*} with the \texttt{align*} environment.
\end{itemize}

\small

\begin{Shaded}
\begin{Highlighting}[]
\KeywordTok{\textbackslash{}begin}\NormalTok{\{}\ExtensionTok{align*}\NormalTok{\}}
\SpecialStringTok{(x+1)\^{}3 \&= (x+1)(x+1)(x+1) }\SpecialCharTok{\textbackslash{}\textbackslash{}}
\SpecialStringTok{        \&= (x+1)(x\^{}2 + 2x + 1) }\SpecialCharTok{\textbackslash{}\textbackslash{}}
\SpecialStringTok{        \&= x\^{}3 + 3x\^{}2 + 3x + 1}
\KeywordTok{\textbackslash{}end}\NormalTok{\{}\ExtensionTok{align*}\NormalTok{\}}
\end{Highlighting}
\end{Shaded}

\normalsize

\begin{itemize}
\item
  An ampersand \framebox{\texttt{\&}} separates the left column (before
  the `\(=\)') from the right column (after the `\(=\)').
\item
  A double backslash \framebox{\texttt{\textbackslash\textbackslash}}
  starts a new line.
\end{itemize}
\end{frame}

\begin{frame}{Exercise}
\protect\hypertarget{exercise-1}{}
\begin{varblock}{Exercise 2 (Maths)}
Typeset the following paragraph in \LaTeX :

\begin{quote}
Let \(X_1, X_2, \dots, X_n\) be a sequence of independent and
identically distributed random variables with mean \(\mu\) and variance
\(\sigma^2 < \infty\), and let \begin{equation}
S_n = \frac{1}{n}\sum_{i=1}^{n} X_i
\end{equation} denote their mean. Then as \(n\) approaches infinity, the
random variables \(\sqrt{n}(S_n - \mu)\) converge in distribution to a
normal \(N(0, \sigma^2)\).
\end{quote}

Click
\href{https://www.overleaf.com/docs?snip_uri=https://raw.github.com//haziqj/learn-latex/main/exercises/02-mathematics/02-mathematics.tex&splash=none}{\beamerbutton{Exercise 2}}
to open this exercise in \textbf{Overleaf}.

\end{varblock}
\end{frame}

\hypertarget{wrap-up}{%
\section{Wrap-up}\label{wrap-up}}

\begin{frame}{Wrap-up}
\begin{center}
\begin{tikzpicture}
\node[anchor=south west,inner sep=0] (image) at (0,0) {\includegraphics[width=0.7\textheight]{figure/confetti.png}};
\node[align=center,black,font={\Huge}] at (image.center) {CONGRATS!};
\end{tikzpicture}
\end{center}

You have now learned how to\ldots{}

\begin{itemize}
\tightlist
\item
  Typeset text in \LaTeX
\item
  Use lots of different commands
\item
  Typeset some beautiful mathematics
\item
  Use several different environments (figures, tables, lists).
\item
  Load packages
\end{itemize}

Next time, we'll see how to use \LaTeX to write structured documents
with sections, cross references, figures, tables and bibliographies.
\end{frame}




\end{document}		